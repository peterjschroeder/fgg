\chapter{HIRELINGS AND HENCHMEN}

\begin{multicols}{2}

\section{HIRELINGS}

\index{Hirelings}Few characters are able to pursue a class.  Those exceptional enough to become a true fighter or wizard are a vast minority in the world, and those who use their skill to become adventurers are more rare still (only about 1\% of the world's population).  On the other hand, due to lifelong training and expertise, the average NPC can be knowledgeable in an assortment of skills that PCs generally lack.  They can be hired for their services, and if well treated, become loyal retainers to the PCs.  Hirelings generally refuse to go on adventures, but they'll accompany PCs and wait at safe locations, such as a camp outside a dungeon.

\subsection{0-LEVEL CHARACTERS}

\index{0-level Characters}The average person is a 0-level (zero-level) character.  They perform their daily jobs, but due to the mundane nature of their vocations, do not gain experience.  These average people make up the backbone of society.  Normally, it is not necessary to know their attributes, but they can be calculated as needed.

\paragraph{Ability Scores:} \index{Ability Scores!of 0-level Characters}Ability scores range from 3--18.  Racial modifiers to the six basic scores don't apply, but racial modifiers to AC and combat do.

\paragraph{Skills:} \index{Combat Skills!of 0-level Characters}\index{Non-combat Skills!of 0-level Characters}A 0-level character is skilled in the use of one weapon, usually a tool they use to perform their jobs (sickles for farmers, hammers for blacksmiths, etc.).  Professional soldiers may be trained in the use of multiple ranged and melee weapons.  Skilled workers are skilled in whatever profession they might have.  

\paragraph{Hit Points:} HP for 0-level characters varies based on profession.

\paragraph{THACO:} \index{THACO!of 0-level Characters}A 0-level character's THACO is always 20.
 
\paragraph{Saving Throw:} \index{Saving Throws!of 0-level Characters}0-level characters have the following saves.  

\noindent
\begin{minipage}{\columnwidth}

\captionof{table}{0-Level Character Hit Points}\label{zerolevelhp}
\noindent
\begin{tabular}{|p{.6\columnwidth}|p{.3\columnwidth}|}
\hline
Profession	& Hit Points \\
\hline\hline
\rowcolor[gray]{.9}Manual Laborer	& 1d8 \\
Soldier	& 1d8~+~1 \\
\rowcolor[gray]{.9}Craftsman	& 1d6 \\
Scholar	& 1d3 \\
\rowcolor[gray]{.9}Invalid	& 1d4 \\
Child	& 1d2 \\
\rowcolor[gray]{.9}Teenager	& 1d6 \\
\hline
\end{tabular}

\end{minipage}

\section{EXPERT HIRELINGS}

\index{Hirelings!Expert Hirelings}Experts are encountered considerably less frequently than commoners and fall into four categories---assassins, spies, sages, and soldiers.  PCs can seek their employ, but their fees tend to be exorbitant and finding them can be an adventure in itself.  Due to the unusual nature of their services, it is advisable to determine their attributes in advance.

\subsection{ASSASSINS}

\index{Hirelings!Assassins}An assassin isn't necessarily an occupation but a mindset.  In game terms, an assassin refers to anybody who's willing to kill in exchange for money.  In lawful and good aligned societies, assassins are extremely difficult, if not  impossible, to find.  In cities of ill repute, finding an assassin could be as easy as walking into a seedy tavern, although finding reputable ones in such a place is much more difficult.  Regardless of location, no assassin is foolish enough to advertise openly.

Hiring an assassin, regardless of the task, is neither a good or lawful act.  Assassins only target persons; they refuse to fight monsters or go on adventures.  The asking price is completely up to the GM; the more powerful the target, the larger the asking price.  An assassin will also ask for more pay if he's expected to use his own resources.
The PCs will explain how the assassin is to perform his task and may attempt to negotiate the price by supplying critical information and/or special equipment.
 
The chance for success of an assassination attempt is up to the GM.  In general, it's best to allow an assassination to succeed if it doesn't slow down the game or result in the PCs easily overcoming a planned adventure.  If a target for assassination is an important NPC, the assassination attempt should fail or be allowed to succeed but with a reasonable downside.  Perhaps the assassination of the Lord Mayor was successful and has drawn the attention of smugglers that were bribing him---exactly what the PCs were hoping for.  The assassination of a powerful priest could fail, resulting in the clergy magically interrogating the assassin and revealing the PCs involvement.  Player characters should never be allowed to hire assassins against each other.

Overuse of assassins should come with additional drawbacks.  More often than not, assassins are evil and dishonest.  There's little sense of client confidentiality and they have no qualms against accepting a better offer, even if it means targeting their former employers. 

\end{multicols}

\noindent
\begin{minipage}{\columnwidth}

\captionof{table}{0-Level Character Saving Throws}\label{zerolevelsaves}
\noindent
\begin{tabular}{|m{0.12\columnwidth}|m{0.148\columnwidth}|m{0.148\columnwidth}|m{0.148\columnwidth}|m{0.148\columnwidth}|m{0.148\columnwidth}|}
\hline
Level	& Paralyzation, Poison, or Death	& Rod, Staff, or Wand	& Petrification or Polymorph	& Breath Weapon	& Spell \\
\hline\hline
\rowcolor[gray]{.9}0 & 16	& 18	& 17	& 20	& 19 \\
\hline
\end{tabular}

\end{minipage}

\begin{multicols}{2} 

\subsection{SPIES}

\index{Hirelings!Spies}A spy is someone who purposefully befriends a group in order to betray them through subterfuge, sabotage, theft, or selling information.  While usually less reprehensible than assassins, spying is a dishonest activity.  As a result, a spy's motives and credibility is always questionable.

As with the assassin, hiring a spy could become an adventure in itself, as advertising goes against the nature of their profession.  A spy's asking price should reflect the spy's reaction to the PCs, and the task at hand.  The spy charges extra, if he's expected to use personal resources on his task.  Again the PCs are expected to haggle by offering critical information and/or special equipment.

The success rate for attempting to spy is left to the GM's discretion and varies based on the task.  In general, spying attempts should be successful, unless they slow down the game or result in a major adventure being completely bypassed.  However, even if a spy is successful, the information may not be wholly accurate, or task may not be complete. 

Few spies have any sense of client confidentiality, and a captured spy is likely to reveal his employers.  Spies are not known for their loyalty and will change sides if a better offer comes along.  Double agents are common, and it's possible that both the PCs and the target become victims of a spy's double cross. 

\noindent
\begin{minipage}{\columnwidth}

\captionof{table}{Sage Knowledge}\label{sageknowledge}
\noindent
\begin{tabular}{|p{.25\columnwidth}|p{.15\columnwidth}|p{.45\columnwidth}|}
\hline
Study			& Chance	& Abilities/Limitations \\
\hline\hline
\rowcolor[gray]{.9}Alchemy			& 10\%	& Brew poisons and acids \\
Architecture	& 5\%	& Specific race or culture \\
\rowcolor[gray]{.9}Art				& 20\%	& Specific race or culture \\
Astrology		& 10\%	& Knowledge in navigation and astrology \\
\rowcolor[gray]{.9}Astronomy		& 20\%	& Knowledge in navigation and astronomy \\
Botany			& 25\%	& Study of plants \\
\rowcolor[gray]{.9}Cartography		& 10\%	& Study of maps \\
Chemistry		& 5\%	& Brew poisons and acids \\
\rowcolor[gray]{.9}Cryptography	& 5\%	&  \\
Engineering		& 30\%	&  \\
\rowcolor[gray]{.9}Folklore		& 25\%	& Specific race/region \\
Genealogy		& 25\%	& Specific race/region \\
\rowcolor[gray]{.9}Geography		& 10\%	&  \\
Geology			& 15\%	& Knowledge of mining \\
\rowcolor[gray]{.9}Heraldry		& 30\%	&  \\
History			& 30\%	& Specific race/region \\
\rowcolor[gray]{.9}Languages		& 40\%	& Specific language \\
Law				& 35\%	&  \\
\rowcolor[gray]{.9}Mathematics		& 20\%	&  \\
Medicine		& 10\%	&  \\
\rowcolor[gray]{.9}Metaphysics		& 5\%	& Specific plane of existence \\
Meteorology		& 20\%	&  \\
\rowcolor[gray]{.9}Music			& 30\%	& Specific race or culture \\
Myconology		& 20\%	& Knowledge of fungi \\
\rowcolor[gray]{.9}Oceanography	& 15\%	&  \\
Philosophy		& 25\%	& Specific race or culture \\
\rowcolor[gray]{.9}Physics			& 10\%	&  \\
Sociology		& 40\%	& Specific race or region \\
\rowcolor[gray]{.9}Theology		& 25\%	& Specific region \\
Zoology			& 20\%	&  \\
\hline
\end{tabular}

\end{minipage}

\subsection{SAGES}

\index{Hirelings!Sages}Sages are professional scholars and professors who boast impressive knowledge in a single field of study.  Sages only reside in large towns and cities, or other area where travelers are frequent; usually within the walls of a university, monastery or church, where there's access to research materials.  If it's not for certain whether a sage with specific knowledge the PCs need resides in a particular town or city, a random roll can be made to determine if one's available.

\paragraph{Chance:} the base percent chance that a particular sage is available.  The GM may apply any bonuses (for such things as size and prominence of the city), or penalties (for such things as low population or extended distance to established land or sea trade routes).  

\paragraph{Abilities/Limitations:} Some sages have special abilities like brewing potions.  Some sage's knowledge is limited to a specific race, region, or culture.  If nothing is listed, the sage's knowledge includes all closely related subjects and topics.
 
\subsection{HANDLING SAGES}

A sage's ability should directly reflect the need to know of the players.  If the PCs are completely stumped and hire a sage for help, the answer could be answered in full, but if the answer to a question would totally bypass or slow down an adventure, the sage should only be able to gather bits and pieces of (still useful) information.  

The GM can also roll randomly to determine if the sage knows the answer to a question in his field.  A sage's ability in his field is equal to 14~+~1d6.  Roll a 1d20 proficiency roll---if the check is equal to or less than the sage's ability, the answer is known, otherwise the sage fails to provide a useful answer.  On a roll of 20, the sage unknowingly gives out false or misleading information.

Questions should be categorized as general (``What creatures live in the forest?"), specific (``Do wood elves live in the forest?"), or exact (``Who's the leader of the wood elves in the forest?").  The type of question asked modifies the 1d20 proficiency roll to answer the question.

As stated previously, sages require access to a library, university, monastery, or church to conduct their research.  If a personal library is used, it must be at least 200 square feet with rare and exotic manuscripts worth at least 1,000 gp per book.  Institutions and private libraries commonly charge annual membership fees (thousands of gold pieces) or charge per day of research (around 100 gp per day).  The PCs are expected to pay this cost as part of the sage's expenses.  Having poorly stocked resources or no resources at all applies penalties.

Sages require time to answer questions.  PCs can rush the sage's work.  Rushing a sage shifts the time required for research to the next lower category (the time for general questions is halved or 1d3 hours), but doing so applies a penalty to his roll.

\noindent
\begin{minipage}{\columnwidth}

\captionof{table}{Sage Proficiency Modifiers}\label{sageproficiencymods}
\noindent
\begin{tabular}{|p{.6\columnwidth}|p{.3\columnwidth}|}
\hline
Type	& Modifier \\
\rowcolor[gray]{.9}Question Type:	& \\
\rowcolor[gray]{.9}\hspace{2em}General	& -- \\
\hspace{2em}Specific	& $-2$ \\
\rowcolor[gray]{.9}\hspace{2em}Exact	& $-4$ \\
Resources:	& \\
\hspace{2em}Complete	& -- \\
\rowcolor[gray]{.9}\hspace{2em}Partial	& $-2$ \\
\hspace{2em}Nonexistent	& $-6$ \\
\rowcolor[gray]{.9}Sage is rushed	& $-4$ \\
\hline
\end{tabular}

\end{minipage}

\noindent
\begin{minipage}{\columnwidth}

\captionof{table}{Research Time}\label{researchtime}
\noindent
\begin{tabular}{|p{.45\columnwidth}|p{.45\columnwidth}|}
\hline
Type	& Time \\
\hline\hline
\rowcolor[gray]{.9}General	& 1d6 hours \\
Specific	& 1d6 days \\
\rowcolor[gray]{.9}Exacting	& 3d10 days \\
\hline
\end{tabular}

\end{minipage}

\subsection{SOLDIERS}

\index{Hirelings!Soldiers}Soldiers are experts trained in warfare, and unlike other hirelings, they're expected to fight and defend the PCs.  Soldiers are 0-level characters, but participate on adventures.  Soldiers provide their equipment but require food and lodging.  Like all hirelings, they expect prompt payment, fair treatment, and will abandon cruel masters.

\noindent
\begin{minipage}{\columnwidth}

\captionof{table}{Military Occupations}\label{militaryoccupations}
\noindent
\begin{tabular}{|p{.6\columnwidth}|p{.3\columnwidth}|}
\hline
Title	& Monthly Wage \\
\hline\hline
\rowcolor[gray]{.9}Archer	& 4 gp \\
Arquebusier	& 6 gp \\
\rowcolor[gray]{.9}Artillerist	& 4 gp \\
Bowman, mounted	& 4 gp \\
\rowcolor[gray]{.9}Cavalry, heavy	& 10 gp \\
Cavalry, light	& 4 gp \\
\rowcolor[gray]{.9}Cavalry, medium	& 6 gp \\
Crossbowman, heavy	& 3 gp \\
\rowcolor[gray]{.9}Crossbowman, light	& 2 gp \\
Crossbowman, mounted	& 4 gp \\
\rowcolor[gray]{.9}Engineer	& 150 gp \\
Footman, heavy	& 2 gp \\
\rowcolor[gray]{.9}Footman, irregular	& 5 sp \\
Footman, light	& 1 gp \\
\rowcolor[gray]{.9}Footman, militia	& 5 sp \\
Longbow man	& 8 gp \\
\rowcolor[gray]{.9}Marine	& 3 gp \\
Sapper	& 1 gp \\
\rowcolor[gray]{.9}Shield bearer	& 5 sp \\
\hline
\end{tabular}

\end{minipage}

\paragraph{Archer:} Foot soldier armed with short bow, arrows, short sword, and leather armor.

\paragraph{Arquebusier:}  An arquebusier is a ranged soldier who's trained with the arquebus.  Arquebusiers wear a wide variety of armors and fight with short swords.

\paragraph{Artillerist:} These specialists are rarely armed.  Their task is to direct and maintain siege weapons.

\paragraph{Bowman, mounted:} These warriors ride light warhorses and wield short bows, long swords or scimitars, and wear leather armor.  Heavier soldiers ride heavy warhorses and wear chain mail.

\paragraph{Cavalry, heavy:} These warriors are mounted on heavy warhorses with barding, wield heavy lances, long swords, maces, and wear plate or field plate armor and large shields.

\paragraph{Cavalry, light:} These warriors ride light warhorses, wear padded armor and small shields, and fight with javelins and swords.  Some carry missile weapons, usually short bows, to fall back on.

\paragraph{Calvary, medium:} These warriors ride light warhorses, wield lances, long swords, maces, and wear scale or banded armor and medium shields.

\paragraph{Crossbowman, heavy:}  These ranged attackers are usually used for defensive positions or during sieges.  They wield heavy crossbows, short swords, daggers, and wear chain mail.  A shield bearer is usually supplied to each man for extra defense while they reload.

\paragraph{Crossbowman, light:} These unarmored ranged warriors wield light crossbows, short swords, and daggers.  

\paragraph{Crossbowman, mounted:}  These unarmored warriors ride light warhorses and wield light crossbows, short swords, and daggers.

\paragraph{Engineer:}  Engineers are specialists who don't fight in battle.  Their job is to supervise siege operations, mine castle walls, fill or drain moats, repair damaged structures, construct and repair siege engines, build bridges, or set up camps.  Some engineers act as sappers or demolitionists.

\paragraph{Footman, heavy:}  These warriors are armed with at least chain mail armor, large shields, and any weapons they prefer.

\paragraph{Footman, irregular:}  These warriors are tribesmen or locals with combat experience, but little discipline or formation.  They're usually unarmored and wield their weapon of choice.  A traditional barbarian, for example, is unarmored and wields a shield, battle axe, or sword.

\paragraph{Footman, light:}  These cheap foot soldiers are usually trained soldiers with little actual experience.  They're usually unarmored, or only lightly so (padded and leather), and wield cheaply manufactured weapons (usually pole arms and maces).

\paragraph{Footman, militia:}  These are conscripted townsfolk and peasants.  They wield whatever equipment they might own, meaning they're usually unarmored and fight with cheap weapons such as clubs, daggers, and pole arms.

\paragraph{Longbow man:}  These highly trained archers wear padded or leather armor and fight with long bows, short swords, or dirks.

\paragraph{Marines:} These are heavy footmen who serve aboard ships.

\paragraph{Sapper:} These light infantry soldiers work under engineers providing labor, fieldwork, and supporting siege operations.

\paragraph{Shield bearer:}  These light infantry soldiers wield tower shields that they set up for stationary archers to use as cover.  In the case of archers being attacked, the shield bearer is expected to cover their retreat.

\subsection{HIRING HIRELINGS}

In order to acquire hirelings, one must first find them.  Commoners are plentiful, but in many cases, the average laymen might be considered the property of a local lord or ruler.  A ruler may levy a tax or demand tribute, if someone else employs his workers.  Depopulating an area or creating a labor vacuum by taking on too many hirelings won't go unnoticed even by the kindest rulers.  

Skilled laborers and trained soldiers are more difficult to find.  Because of their nature, spies, assassins, and sages are very rare and more complicated to locate, requiring the PCs to have the right connections or ask the right questions (which could lead to trouble).  Trained soldiers are almost always in the employ of someone's private army, however, mercenary guild houses train and give shelter to new soldiers.  Adventuring is a dangerous occupation, and soldiers, in particular, are less likely to work for someone with a reputation of returning without the men they hired.

\subsection{WEEKLY WAGE}

\index{Hirelings!Wages}Hirelings demand fair payment.  This amount doesn't include room and board, food, or traveling expenses, which they expect to be equal to what they're normally accustomed to.  Hirelings may demand more money if their prospective employers have a reputation for employees winding up dead.

\noindent
\begin{minipage}{\columnwidth}

\captionof{table}{Common Wages}\label{commonwages}
\noindent
\begin{tabular}{|m{.45\columnwidth}|m{.2\columnwidth}|m{.2\columnwidth}|}
\hline
Profession	& Weekly Wage	& Monthly Wage \\
\hline\hline
\rowcolor[gray]{.9}Clerk		& 2 gp	& 8 gp \\
Stonemason	& 1 gp	& 4 gp \\
\rowcolor[gray]{.9}Laborer		& 1 sp	& 1 gp \\
Carpenter	& 1 gp	& 5 gp \\
\rowcolor[gray]{.9}Groom		& 2 sp	& 1 gp \\
Huntsman	& 2 gp	& 10 gp \\
\rowcolor[gray]{.9}Ambassador/official	& 50--150 gp	& 200--600 gp \\
Architect	& 50 gp	& 200 gp \\
\hline
\end{tabular}

\end{minipage}

\section{HENCHMEN}

\index{Henchmen}A henchman starts as a GM-run NPC with a character class.  They're hired by the PCs or willingly join their group to participate on adventures.  Henchmen ask for fair treatment and a share of any treasure found.  Unlike hirelings, henchmen gain experience for their exploits and level up as PCs do.  Because they're not PCs, henchmen only gain half experience for their efforts.

If a PC befriends a henchman to the point where they regard each other as close friends or allies, the GM can present an option for the henchman to become a permanent PC for that player.  If the player agrees, he assumes control of that henchman and plays it as he would his original PC.  The GM hands over the character sheet and all relevant information, but some secrets or special abilities, can still be kept secret.  A befriended henchman still counts toward the PCs limit.  This count is permanent, even if the henchman dies.

Although the PC controls the henchman, the character still has a mind of his own.  Henchmen don't give out their own treasure unless necessary, don't like being cheated or lied to, and don't take the blame for something they didn't do.  The GM has the right to step in and dictate what the henchman can or cannot do.  Players abusing henchman or acting out of character should be corrected or else the henchman leaves (and usually spreads bad news about the PC).  Henchmen are still expected to be treated fairly and receive their share of treasure.  If their friend dies or is captured, the henchman will leave or stick around long enough to raise or rescue the character.  Henchman can also be used as replacement characters if a PC dies and the player doesn't want to make a new character.

Henchman should always be at least one level below the PC they befriend.  If a henchman ever meets or exceeds the PC's level, he temporarily leaves to seek his own fortune.  He may return at a later time (when the PCs reach his level) or act as a major player in future events.  The GM regains control and can play the henchman as he sees fit, even returning the character back to the PC at a later point.

\section{HIRING SPELL-CASTERS}

There may come a time when the PCs require the services of spell-casters.  Like expert hirelings, finding a spell-caster may be difficult.  Wizards can be found all over from secluded hedge wizards high in the mountains to court wizards in a king's castle.  Priests are easier to find, however, they're pickier than wizards.  A wizard or bard is satisfied with rare components or money, but priests require both a donation and devotion to their cause.  Neutral priests are usually satisfied with monetary donations but good or evil priests are likely to refuse service to PCs of the opposite alignment.

The cost of casting spells in the below table is a general price and the GM should feel free to change it.  For wizards, these prices also reflect the chance to copy a spell from a spell book and can be used as an example of what a typical scroll might cost someone.

\section{MORALE}\index{Morale!of Henchmen and Hirelings}

Henchmen and hirelings, like other NPCs, have a morale rating.  Even those loyal to the PCs are still free willed characters and in most cases, do not harbor suicidal devotion for their employers.  The base morale for henchmen is 15 and for hirelings its 12.  Several modifiers may affect the base morale.  Also refer to the sections regarding morale at the end of chapter 7 for additional modifiers, conditions that may force a morale check, etc.

\noindent
\begin{minipage}{\columnwidth}

\captionof{table}{Spell-Caster Services}\label{spellservices}
\noindent
\begin{tabular}{|p{.4\columnwidth}|p{.5\columnwidth}|}
\hline
Spell	& Minimum Cost \\
\hline\hline
\rowcolor[gray]{.9}\textit{Astral spell}	& 2,000 gp per person \\
\textit{Atonement}	& * \\
\rowcolor[gray]{.9}\textit{Augury}	& 200 gp \\
\textit{Bless}	& * \\
\rowcolor[gray]{.9}\textit{Charm}	& 1,000 gp \\
\textit{Clairvoyance}	& 50 gp per caster level \\
\rowcolor[gray]{.9}\textit{Commune}	& * \\
\textit{Comprehend languages}	& 50 gp \\
\rowcolor[gray]{.9}\textit{Contact other plane}	& 5,000 gp~+~1,000 gp per question \\
\textit{Continual light}	& 1,000 gp \\
\rowcolor[gray]{.9}\textit{Control weather}	& 20,000 gp \\
\textit{Cure blindness}	& 500 gp \\
\rowcolor[gray]{.9}\textit{Cure disease}	& 10 gp per point healed \\
\textit{Cure light wounds}	& 10 gp per point healed \\
\rowcolor[gray]{.9}\textit{Cure serious wounds}	& 20 gp per point healed \\
\textit{Cure critical wounds}	& 40 gp per point healed \\
\rowcolor[gray]{.9}\textit{Detection spells (any)}	& 100 gp \\
\textit{Dispel magic}	& 100 gp per caster level \\
\rowcolor[gray]{.9}\textit{Divination}	& 500 gp \\
\textit{Earthquake}	& * \\
\rowcolor[gray]{.9}\textit{Enchant an Item}	& 20,000 gp plus other spells \\
\textit{ESP}	& 500 gp \\
\rowcolor[gray]{.9}\textit{Explosive runes}	& 1,000 gp \\
\textit{Find the path}	& 1,000 gp \\
\rowcolor[gray]{.9}\textit{Fire trap}	& 500 gp \\
\textit{Fools' gold}	& 100 gp \\
\rowcolor[gray]{.9}\textit{Gate}	& * \\
\textit{Glyph of warding}	& 100 gp per caster level \\
\rowcolor[gray]{.9}\textit{Heal}	& 50 gp per point healed \\
\textit{Identify}	& 1,000 gp per item or function \\
\rowcolor[gray]{.9}\textit{Invisible stalker}	& 5,000 gp \\
\textit{Invisibility}	& 500 gp \\
\rowcolor[gray]{.9}\textit{Legend Lore}	& 1,000 gp \\
\textit{Limited wish}	& 20,000 gp** \\
\rowcolor[gray]{.9}\textit{Magic mouth}	& 300 gp \\
\textit{Mass charm}	& 5,000 gp \\
\rowcolor[gray]{.9}\textit{Neutralize poison}	& 100 gp \\
\textit{Permanency}	& 20,000 gp** \\
\rowcolor[gray]{.9}\textit{Plane shift}	& * \\
\textit{Prayer}	& * \\
\rowcolor[gray]{.9}\textit{Protection from evil}	& 20 gp per caster level \\
\textit{Raise dead}	& * \\
\rowcolor[gray]{.9}\textit{Read magic}	& 200 gp \\
\textit{Regenerate}	& 20,000 gp \\
\rowcolor[gray]{.9}\textit{Reincarnate}	& * \\
\textit{Remove curse}	& 100 gp per caster level \\
\rowcolor[gray]{.9}\textit{Restoration}	& * \\
\textit{Slow poison}	& 50 gp \\
\rowcolor[gray]{.9}\textit{Speak with dead}	& 100 gp per caster level \\
\textit{Suggestion}	& 600 gp \\
\rowcolor[gray]{.9}\textit{Symbol}	& 1,000 gp per caster level \\
\textit{Teleport}	& 2,000 gp per person \\
\rowcolor[gray]{.9}\textit{Tongues}	& 100 gp \\
\textit{Wish}	& 50,000 gp** \\
\rowcolor[gray]{.9}\textit{Wizard lock}	& 50 gp per caster level \\
\hline
\end{tabular}
\noindent\begin{tabular}{p{.95\columnwidth}}
*This spell is normally cast only for those of like faith or alignment. A payment or service may still be required. \\
**An exceptional service will also be required such as a favor or magic item. \\
\end{tabular}\vspace{.5em}

\end{minipage}

\noindent
\begin{minipage}{\columnwidth}

\captionof{table}{Permanent Morale Modifiers}\label{permanentmoralemodifiers}
\noindent
\begin{tabular}{|p{.7\columnwidth}|p{.2\columnwidth}|}
\hline
Factor	& Modifier \\
\hline\hline
\rowcolor[gray]{.9}NPC is lawful*	& +1 \\
NPC is good	& +1 \\
\rowcolor[gray]{.9}NPC is evil	& $-1$ \\
NPC is chaotic*	& $-1$ \\
\rowcolor[gray]{.9}NPC is of different race	& $-1$ \\
NPC has been with PC for 1 year or more	& +2 \\
\hline
\end{tabular}
\noindent\begin{tabular}{p{.95\columnwidth}}
*These modifiers appear on table \ref{moralemods}.  Don't apply them twice. \\
\end{tabular}\vspace{.5em}

\end{minipage}

\end{multicols}



