\chapter{NON-COMBAT SKILLS}

\begin{multicols}{2}

\index{Non-combat Skills}Skills help define a character's abilities beyond their class.  Skills take the form of either an ``ability score check" or a ``skill check" made when appropriate.  Skills are purchased during character creation and when acquiring levels.  A character can only use skills they're proficient in: a character is either proficient in a skill or they're not (untrained).

\index{Ability Check}A check is a d20 roll plus or minus modifiers against the character's relevant ability score.  If the character's roll is less than or equal to their their ability score they succeed.  Positive modifiers increase the ability score (making things easier) and negative modifiers reduce it (making things more difficult).  A roll of 1 is always a success and a roll of 20 is always a failure regardless of the ability score.  

As a rule, a check is only made when the results are important.  Most tasks are so mundane that the GM should consider it an automatic success.  A check may be requested to see how well the character has succeeded or failed.  A greater margin of success (or failure) should lead to more dramatic results.

\section{LEARNING SKILLS}

\index{Non-combat Skills!Learning Skills}Characters begin with skill points at level 1, including a bonus granted by intelligence, and gain new skills at certain levels.  Skill points at 1\textsuperscript{st} level must be spent during character creation.  Spending the required skill points is enough to become proficient in that skill.  Investing an additional point in a skill gives a character the ``Advanced" status, granting a +3 bonus.  Investing another point gives a character ``Mastery" which grants another +3 bonus (+6 max).  

At further levels, skill points may be saved but skill points can only ever be spent while leveling up.  Learning new skills assumes the players are training in their downtime.  If the GM requires it, characters may need to find a trainer to learn a particularly difficult skill.

\noindent
\begin{minipage}{\columnwidth}

\captionof{table}{Skill Points by Class}\label{classNCSP}
\noindent
\begin{tabular}{|m{0.18\columnwidth}|m{0.30\columnwidth}|m{0.37\columnwidth}|}
\hline
Class	& Starting Skills		& New Skills \\
\hline\hline
\rowcolor[gray]{.9}Warrior	& 3	& 1 every 3\textsuperscript{rd} level \\
Wizard	& 4	& 1 every 3\textsuperscript{rd} level \\
\rowcolor[gray]{.9}Priest	& 4	& 1 every 3\textsuperscript{rd} level \\
Rogue	& 3	& 1 every 4\textsuperscript{th} level \\
\hline
\end{tabular}

\end{minipage}

\index{Non-combat Skills!Additional Skills}\subsection{OPTIONAL: ADDITIONAL SKILLS}

Adventurers receive very few skill points.  Over the course of their entire careers, an adventurer will either become moderately proficient in a handful of skills or a master of few.  The reason is because a player character's primary field of specialization is adventuring and the uniqueness of player classes represents this.  If players need a specialist, there are plenty of opportunities to hire one.

If GMs want players to be more self-sufficient and robust in their skills they may offer additional skill points (perhaps the starting value every 3\textsuperscript{rd} or 4\textsuperscript{th} level).  This means players will rely less on outside help but it also means there will be overlap where characters will share equal proficiency with each other.  The rule that only the most proficient character is tested should still apply.  

\subsection{BEGINNING TRADE OR KNOWLEDGE}

Before a character began adventuring, they likely apprenticed in a trade or specialized in a field of study.  Upon creation a character receives a free trade or knowledge skill of their choice that they're proficient in.  Trade and knowledge skills are detailed in the skill descriptions.

\subsection{WHAT DOES A CHARACTER KNOW?}

Being proficient in a skill is enough to use it but doesn't necessarily represent their experience.  A person with high charisma may be great at persuading people but not exactly because they know psychology or know the right words to say like a more proficient person would.  Think of this as the difference between talent and experience.  A character with advanced or master proficiency in a skill is more knowledgeable than a person without even if their ability score is weaker.  

\section{USING SKILLS}

\index{Non-combat Skills!Using Skills}Unless otherwise noted, using a skill or making a check is a single action that takes a full round.  Sometimes a skill may require more or less time as appropriate. A carpenter building a wooden table may take a full day while building a simple cottage would require a month or more.  

Because player characters are often in a group, the character with the best skill check should be used when group actions are taken.  The results of a skill are either success or failure.  A failed roll may allow a reroll given the situation but generally a failed check carries a dramatic result.  For example, failing on a climb check results in you falling.  Failing to talk down a merchant may allow renegotiations but he's now angry and even more difficult to talk down than before.  

It's assumed the character using the skill has the appropriate tools or means to use the skill.  You can't chop down a tree without a good cutting tool and being blindfolded makes it impossible to see.

\subsection{ABILITY SCORE CHECKS}

\index{Ability Check}Ability score checks (or simply ability checks) are rolls made against the related ability score.  Ability checks are tested in one of two scenarios:

\paragraph{Untrained Skill:} If a character wants to use a skill untrained they must test against their ability score.  The difficulty of performing a check untrained is always at a $-2$ penalty plus any negative penalties resulting from the skill's inherent difficulty.  If a skill has a bonus it is not applied and if a skill allows abilities to be made without a check, an untrained character must instead check for them.

Some skills cannot be used untrained and are designated in their description.

\paragraph{General Use:} Some actions and situations are so general that they do not need to be attributed to a skill.  Trying to spot someone in a crowd or maintain your balance on a slippery floor doesn't require a specific skill: anyone can do it.  Whenever a situation would call for a check but no skill is applicable, the GM should assign an ability check.  

\subsection{AIDING ANOTHER}

\index{Non-combat Skills!Aiding Another}Although the character with the best skill in a group should be checked, other characters may contribute as the situation warrants.  Characters proficient in the skill used may assist with a skill check: a successful check results in a +2 modifier.  Only one character may assist at a time.  If any assisting character rolls a dramatic failure a $-2$ penalty is applied instead: the character gets in the way by accident!  

Some skills can't be assisted (assisting someone in a card game may be cheating) or require specific gear (pitons and rope for climbing).  The GM has final say if a skill allows assists.  You can assist in a general ability check if the situation allows for it but assisting in a skill requires you to be proficient.

\subsection{DIFFICULTY MODIFIERS}

Certain scenarios may call for a penalty or bonus to the skill check.  A rusty hammer and cracked forge makes smithing difficult just as rain will make a rocky surface slippery.  Modifiers should be assigned from $-4$ to +4 being impossibly difficult to incredibly easy.

\section{SKILL DESCRIPTIONS}

\textbf{General} skills are skills that anyone can learn regardless of class.  A skill's difficulty is the modifier applied to the character's check.  This doesn't include any situational modifiers the GM may apply.  

\textbf{Untrained (U)} skills can be used untrained.  If a skill doesn't say it can be used untrained, it cannot be used at all.

\index{Animal Handling}\paragraph{Animal Handling (General, Wisdom, $-1$, 2 skill points)}

Animal handling includes the ability to rear and train animals, magical and mundane.  

\textit{Calm Animal}: A successful check calms a frightened or agitated non-magical animal.  This skill does not stop a hungry predator from attacking but does keep an angry bull from charging or a dog from barking.

\textit{Train Animal}: Animal handling allows a person to train normal animals.  Animals can be taught tasks or tricks.  Tasks are general commands such as ``guard," ``attack," or ``fetch."  Tricks are specific tasks such as training a horse to count with its hooves or a dog to balance a treat on its nose.  A particular animal can know a combination of 2d4 tasks and tricks.  Training for a task requires 1d3 months of uninterrupted training while a trick requires 2d6 weeks.  At the end of this period a skill check is made to see if the training is successful.

Magical creatures and monsters with animal intelligence can be trained with a $-4$ penalty.  Such creatures, including wild predators, must be tamed when they're young.  This taming process requires a month of uninterrupted work followed by a skill check.  A successful check means the creature is tamed and further work allows training.  A failed check means the creature retains its instinctual nature and can never be trained.

Special: Rangers and druids spend only 1 skill point to become trained in this skill.  

\index{Acrobatics}\paragraph{Acrobatics (Rogue, Dexterity, +0, 1 skill point)}

Trained acrobatics are experts at balancing and tumbling.  In addition to its normal uses, an acrobat can use this skill in place of a perform skill to earn money.  A character cannot use this skill if they are more than lightly encumbered. 

\index{Armor Class!Effects of Defensive Roll}\textit{Defensive Roll}: By not attacking in a round, a character can improve their AC by 4 during a round.  

\textit{Quick Attack}: When fighting unarmed, a trained acrobat gains a +2 bonus to their attack roll.

\textit{Tuck-and-Roll}: A character who falls from a height of 60 feet or less can reduce their damage by half on a successful check.  A successful check results in no damage from a height less than 20 feet.

\textit{Balance}: A character can maintain their balance on narrow surfaces not inclined greater than 45 degrees up or down.  A character can move up to 60 feet in a round (not exceeding their maximum movement) and must succeed on a skill check each increment of 60 feet.  A surface that's an inch or less (the width of a rope) imposes a $-10$ penalty, $-5$ if the surface is two-to-six inches, and no penalty if the surface is seven inches or wider.  Advanced proficiency reduces penalties by 2 and mastery reduces penalties by another 2 points.  A balancing rod reduces penalties by 2 although strong winds make balancing difficult.  

A character can fight while balancing at a $-5$ penalty to their attack roll and must immediately make a check at the beginning of the round.  A character loses their AC bonus from dexterity and if damaged they must immediately succeed on a check or fall.

\index{Appraise}\paragraph{Appraise (Rogue, Intelligence, +0, 1 skill point)}

A character can assess the value of a valuable objects (antiques, art, jewels, gems, and other crafted mundane objects) including the ability to tell if an item is a counterfeit or fake.  The character must study the item for a full minute and a successful check gives them an accurate assessment with a 10\% margin of error up or down.  A roll of 20 gives a wildly inaccurate reading.  The GM should roll this check in secret.

\index{Blind-Fight}\paragraph{Blind-Fight (Rogue/Warrior, 2 skill points)}

A character trained in this skill isn't hindered by poor lighting.  They suffer no AC penalty in dark conditions and the attack penalties for dim and no lighting are reduced by half ($-1$ and $-2$ respectively).  The penalty to attack invisible creatures is reduced to $-2$ (although the character cannot pinpoint their exact location) and characters move at one-half speed in total darkness vice one-third.

These benefits only apply to melee combat.  Penalties are not reduced for ranged attacks and the $-4$ AC penalty still applies to ranged attacks.  A character with blind-fight cannot actually see in complete darkness, they can just compensate for poor lighting better than most.  A character can only be trained in this skill: further proficiency doesn't give any extra benefits.

\index{Charioteer}\paragraph{Charioteer (General, Dexterity, +0, 1 skill point)}

A trained charioteer automatically improves the speed of chariots by one-third (up to the mount's maximum value).  This skill is checked to maintain control or perform difficult maneuvers.

\textit{Special}: Warriors gain a +2 bonus while using this skill.

\index{Climbing}\paragraph{Climbing (General, Strength, +0, 1 skill point, Untrained)}

Climb is a skill that is explained in Chapter 9: Adventuring and Exploration.  An untrained character's base climb chance is 40\%.  A proficient climber gains a +10\% bonus for each skill point spent (up to +30\% for mastery).  Thieves do not gain a bonus for their thief skill to climb walls: this special thief ability is used when climbing without equipment.

\index{Cross-country Running}\paragraph{Cross-Country Running (Warrior, Constitution, $-6$, 1 skill point)}

This skill allows techniques that aid in moving swiftly over open terrain.  A trained runner may move at twice their movement rate in a day's march (8 hours) and continue this movement on successive days with a skill check.  At the end of a day spent running, the character must get 8 hours of rest or they're immediately fatigued the next day and cannot run until they rest.  

A character who runs cross-country for more than an hour suffers a $-1$ penalty to attack rolls until they rest for the night.

\index{Disguise}\paragraph{Disguise (Rogue, Charisma, $-1$, 1 skill point, Untrained)}

With the aid of cosmetics and clothing, a character can change their appearance.  A character can change their apparent height, up to a foot shorter or taller, using mundane means.  A character can change their race, sex, and age category at a $-2$ penalty each (penalties are cumulative).  

A disguise is a special skill check rolled in secret. The margin of success or failure is noted by the GM.  In general, a disguise is always successful unless an observer has reason to be suspicious.  Any suspicious observer is allowed a wisdom check to see through the disguise at a penalty/bonus based on the disguiser's margin of success/failure respectively (a successful disguise makes it harder to spot and vice versa).  If the disguiser is dressed as someone specific then their disguise is easier to spot by observers familiar with that person.  An observer who recognizes a person on sight gains a +4 bonus, +6 if they're friends or associates, +8 if they're close friends, and +10 if they're intimate.

\index{Endurance}\paragraph{Endurance (Rogue/Warrior, Constitution, +0, 2 skill points)}

This skill allows a character to exert themselves in strenuous situations (running, swimming, holding breath, and so on) twice as long as normal.  Extremely strenuous activities or further stress requires skill checks to succeed.  A trained character still requires the normal amount of food, water, and sleep and function properly. 

\index{Etiquette}\paragraph{Etiquette (General, Charisma, +0, 1 skill point)}

The character is knowledgable in the proper etiquette of dealing with royalty and nobility.  They know how to address people of rank and the proper manners of their culture.  Distinct or unique situations that occur rarely, such as knowing the proper way to greet an extraplanar being or a visit from the dwarven king, requires a skill check not to flub.  The GM should give characters trained in etiquette hints or overlook minor faux pas in roleplay: the character usually knows more than the player in these situations.  However, this skill shouldn't circumvent roleplay completely; just because you know the rules of socializing doesn't save you from being offensive or rude.

\index{Forgery}\paragraph{Forgery (Rogue, Dexterity, $-1$, 1 skill point)}

This skill allows someone to create duplicates of handwritten documents and detect handwritten forgeries.  A person can forge handwritten documents provided they have a reference available.  It's difficult to forge a person's handwriting and a forger takes a $-2$ penalty to forge an autograph and a $-3$ penalty to forge an entire document written by a specific person.  A single check is made for the entire document or 10 pages, whichever is smaller.

On a successful check, the document passes all observation except by those intimately familiar with the handwriting (usually the original writer themselves).  On a failed check, the document is still passable by most people but anyone familiar with the type of document can detect it if they're suspicious.  A check that results in automatic failure is instantly detectable by anyone who handles these kind of documents.  As an example, an officer is intimately familiar with military orders while a rank-and-file soldier is only familiar with their direct superior's handwriting.

A person trained in forgery can detect forgeries on a successful check.  A failed check reveals no information while an automatic failure results in an incorrect conclusion (a forgery is legit and a legitimate document is a forgery).  The GM rolls forgery checks in secret: a character always believes in their own skill even if they fail the check.

A common method to defeat forgeries is signing documents with wax seals or using special materials.  If a forger doesn't have the special materials to recreate a document, it automatically fails any inspection by someone even remotely familiar with the material.

\index{Healing (Proficiency)}\paragraph{Healing (Priest, Wisdom, $-2$, 2 skill points)}

A trained healer can administer advanced care using natural means.  If a wounded character is tended within 10 minutes of being injured, a healer can restore 1d3 hit points with a successful check.  A character can only be healed once per day in this way and cannot recover more hit points than lost in their injuries within a turn.  

A trained healer can oversee 6 patients during long term care.  A person under a healer's care can recover 1 hit point per day even while making strenuous actions.  A character recovers 2 hit points per day under complete rest and a healer that's also proficient in herbalism can heal 3 hit points per day under complete rest.  

A character poisoned with venom (any poison that afflicts via wounding or enters the bloodstream) can be treated by a healer.  If the healer treats the character within the next round of being poisoned and works with them for five rounds total, the poisoned character gains a +2 bonus to their saving throw (delay the saving throw until after preparations).  The treatment must fully take 5 rounds or else the saving throw is made immediately when the work is disrupted.  A healer that's also trained in herbalism can treat all poisons.

A trained healer can diagnose and cure natural diseases.  A full day of treatment is required followed by a skill check.  Success reduces any variable damage and duration of the disease to its minimum although any damage already inflicted isn't healed.  A healer trained in herbalism can diagnose the cause of magical diseases but magic is still required to cure these.

\index{Jumping}\paragraph{Jumping (Rogue, Strength, +0, 1 skill point, Untrained)}

A trained jumper can perform exceptional leaps and vaults without the use of a skill check.  

\textit{Broad Jump}: With a 20 foot running start, a person can leap 2d6~+~level in feet with a maximum distance of six times their height.  Without a running start, a person can leap 1d6~+~half-level in feet.

\textit{Running Vertical Jump}: With a 20 foot running start, a person can leap straight up 1d3~+~half-level in feet with a maximum height of one-and-a-half times their height.  Without a running start, a person can only jump 3 feet.

\textit{Pole Vault}: A character can attempt to vault with a sufficiently long pole.  A pole vault requires a 30 foot running start and a pole that's four to ten feet longer than the character's height.  A pole vault's maximum distance is equal to one-and-a-half times the length of the pole and ends with the character dropping the pole and ending up prone on the ground.  A character can land on their feet if the distance or height of the vault isn't greater than half the pole's length.

For example, a character with a 20 foot long pole can leap through a vertical opening 20 feet in the air (landing prone), leap into an opening 10 feet in the air (landing on their feet), or clear a distance of 30 feet total (landing prone).

Characters untrained in jumping can only jump half as long as a trained jumper: 1d6~+~level running jump, 1d3~+~half-level standing jump, 1~+~half-level running vertical jump, and 1 foot standing jumps.  An untrained character cannot pole vault.

\end{multicols}

\noindent
\begin{minipage}{\columnwidth}

\captionof{table}{Knowledge Skills}\label{knowledgeskills}
\noindent
\begin{tabular}{|p{0.12\textwidth}|p{0.58\textwidth}|p{0.23\textwidth}|}
\hline
Knowledge	& Description	& Notes \\
\hline\hline
\rowcolor[gray]{.9}Ancient History		& Knowledge about the past history of a specific culture, civilization, or race.  Ancient history is far removed and esoteric from what is commonly taught.	& Priest/Rogue/Wizard, Intelligence, $-1$, 1 skill point \\
Animal Lore	& Knowledge about natural animals and their behaviors.  This skill allows the character to discern an animal's disposition on a successful check.  A failed check reveals no information and a failure of 5 or more gives a false reading.  Magical animals can be observed at a penalty based on how uncommon they are.  A trained person may also make an animal call; only calls that are physically possible are allowed (a bird whistle is possible but a convincing lion's roar is difficult for an adult human).  A successful check means the call is convincing and only magical detection reveals that it's false.  A failed check means the call is still convincing but a wisdom check is allowed to detect that it's false and creatures or animals used to the call can automatically detect it.	& Warrior/Priest, Intelligence, +0, 1 skill point \\
\rowcolor[gray]{.9}Astrology	& Proficiency in the stars, heavenly bodies, and their meaning.  Astrologers can predict future events up to 30 days on a successful check.  Events are general and reveal only their potential result.  Astrology grants a +1 bonus to navigation if the stars are visible.	& Priest/Wizard, Intelligence, +0, 2 skill points. \\
Engineering	& Engineers are trained in the design of complex mechanical things from simple machines like bridges and catapults to large buildings such as dams.  This skill only allows the design, not the actual construction of such machines although engineers can supervise their construction.  Engineers can operate complex machinery, plan sieges, and detect flaws in building structures.	& General, Intelligence, $-3$, 2 skill points \\
\rowcolor[gray]{.9}Gaming	& Knowledge of games of chance common and obscure.  The character has deep knowledge of probabilities, tricks, and techniques.  The character can make a proficiency check to cheat but a roll of 17--20, regardless of success, indicates the character is caught.  Characters on a winning streak may be banned outright or accused of cheating even if it's not true.	& Rogue/Warrior, Charisma, +0, 1 skill point \\
Herbalism	& Knowledge of natural plants and herbs.  An herbalist can identify common plants in the wilderness and apply them to the creation of nonmagical mixtures.  Herbalists can also create poisons from natural substances.	& Priest/Rogue/Wizard, Intelligence, $-2$, 2 skill points \\
\rowcolor[gray]{.9}Local History	& The character has distinct and esoteric knowledge of a specific country or region.  He knows common knowledge and facts thought lost to history.  A trained character can tell a story as part of a performance, gaining a +2 bonus to charisma during a social encounter on a successful check.	& Priest/Rogue, Intelligence, +0, 1 skill point \\
Navigation	& This skill encompasses navigating via the wind, stars, and currents and the proper use of navigation equipment such as sextants and gyroscopes.  The chance of getting lost on land is reduced by $-5$\% and $-20$\% at sea.	& General, Intelligence, $-2$, 1 skill point \\
\rowcolor[gray]{.9}Religion	& Deep knowledge of your own religion and religions of neighboring areas.  Extra skill points grants broader knowledge of religions.	& Priest/Wizard, Wisdom, +0, 1 skill point \\
\hline
\end{tabular}

\end{minipage}

\begin{multicols}{2}

\noindent\includegraphics[width=\columnwidth, height=3.75in]{testblock.pdf} 

\index{Knowledge}\paragraph{Knowledge (Varies)}

Knowledge skills allow a character to recite facts about a broad category of knowledge.  One skill point gives proficiency in a single realm of knowledge.  Knowledge checks should be called when the answer to a question is specific and not well known (for example, most people can point out constellations but few know the individual names like an astronomer would).  The more obscure a question is, the less chance the character actually knows it.  A person either knows the answer to a question or they don't: no rerolls are allowed unless time is spent researching the topic.

\index{Languages!Ancient}\paragraph{Languages, Ancient (Priest/Wizard)}

This skill allows a character to learn obscure languages that few creatures still speak daily.  

\index{Languages!Modern}\paragraph{Languages, Modern (General)}

One skill point allows a character to speak a new language provided a teacher is available.  A character can never learn more languages than their intelligence allows.  This skill does not allow a character to become literate in a language (see read/write language) unless the alphabet is derived from common or a language they already know how to speak.

\index{Perform}\paragraph{Perform (Varies)}

Perform is a broad skill that encompasses performance and showmanship skills.  A street or tavern performer in a modestly sized town can earn copper pieces equal to their margin of success each day or more in a city.  Notable performers will eventually be invited to private gatherings.  Legendary performers are noticed by royalty and even extraplanar beings.  A character must be trained in perform skills individually.  

\end{multicols}

\noindent
\begin{minipage}{\columnwidth}

\captionof{table}{Perform Skills}\label{performskills}
\noindent
\begin{tabular}{|p{0.12\textwidth}|p{0.58\textwidth}|p{0.23\textwidth}|}
\hline
Performance	& Description	& Notes \\
\hline\hline
\rowcolor[gray]{.9}Dance	& Knowledge of a broad category of dance types.  Popular dances are easy to perform but more esoteric or cultural dances are more difficult.  	& General, Dexterity, +0, 1 skill point, Untrained \\
Juggling	& The ability to juggle handheld objects.  No skill check is necessary for a basic performance unless objects larger than small size are used.  Rogues may use juggling to catch small thrown weapons (not missiles) aimed at them or within close range such as daggers or darts.  If they succeed on an attack roll using their base THACO (modify for dexterity but no other bonuses) against AC 0, they successfully catch the object.  A failure automatically causes damage even if the attack would have missed.  The rogue must have a free hand and must be aware of the attack.  	& General, Dexterity, $-1$, 1 skill point \\
\rowcolor[gray]{.9}Instrument	& The character is proficient in a single instrument.  Additional skill points allow mastery of a new instrument.  Skill checks aren't required for most performances.	& General, Dexterity, $-1$, 1 skill point \\
Singing	& The character is proficient in vocal performances.  Training is good enough in most cases but singing in an opera or other taxing performances requires a skill check.	& General, Charisma, +0, 1 skill point, Untrained \\
\hline
\end{tabular}

\end{minipage}

\begin{multicols}{2}

\index{Read Lips}\paragraph{Read Lips (Rogue, Intelligence, $-2$, 2 skill points)}

A trained lip reader can discern a conversation via lip movement and body language.  Proficiency allows a character to know one language they can speak.  They must be within 30 feet of the speaker and be able to see them speak.  A successful skill check allows the character to understand the gist of the conversation with basic details and simple names (about 70\% of a conversation).  A failed check gives no result.

\index{Read/Write Languages}\paragraph{Read/Write Language (Rogue/Wizard)}

This skill allows a person to become literate in a language they can speak.  

\index{Riding}\paragraph{Riding (General, Dexterity, +0, 1 skill point, Untrained)}
A trained rider can handle piloting a mounted creature.  Most actions don't require a check unless the action is dangerous and failure could result in injury.  It's assumed a mount is fitted with saddle and reigns.  Riding barebacked is difficult and gives a $-2$ penalty to all checks and any actions that would normally happen without a check must succeed on a skill check.

This skill assumes normal, four legged land-based mounts are ridden (horses, camels, donkeys, etc.).  Exotic mounts, such as monstrous creatures, flying creatures, and underwater creatures all impose a $-2$ penalty to ride checks due to their difficulty.

\textit{Saddle Vault}: A trained rider can mount the saddle of a stationary creature as part of their movement.  A skill check is necessary to instantly mount a creature without a saddle, to get the creature to move in the same round it was mounted, or to mount a moving creature.

\textit{Jump}: A trained rider can make a mounted creature jump small obstacles or gaps.  With a running start, a four legged creature with a rider can easily leap obstacles no more than half their height or gaps that are less than twice their height in distance.  A successful check can make the creature double its jumping height and triple its jumping length.  A failed check causes the mount to rear and the rider must make another check to remain in the saddle.

\textit{Spur}: On a successful check a rider may increase their mount's speed by 6.  This speed can be maintained for up to four rounds but a successful check is required to maintain it.  If a check fails on a subsequent round the mount's speed is reduced by half for one turn.  A mount's speed is reduced by 6 or half (whichever is less) for one turn after being spurred on.

\textit{Guide with Knees}: A trained rider can automatically guide a mount with their knees, freeing both hands.  A check is needed to stay in the saddle if the rider suffers from damage.  If the rider has at least one hand on the reigns they don't have to make a check if injured.

\textit{Shield}: A trained rider can drop down the side of their mount, using its body as a shield.  The rider cannot attack in the same round while using this action.  The character's armor class is improved by 6 and any attacks that would hit them will hit the mount instead.  No check is required to use this maneuver unless the character is armored or encumbered: a penalty of $-1$ is imposed for every 1 point of AC for the rider's armor and a $-4$ penalty if their movement is reduced by carried weight.

\textit{Leap from Saddle}: A trained rider can leap from their saddle and attack one creature within 10 feet.  The rider can jump at any point during the mount's movement on a successful check at a $-4$ penalty.  A failed check causes the rider to fall, suffering 1d3 points of damage.  Only one melee attack can be used in a round where the rider leaps from the saddle.

\index{Set Snares and Traps}\paragraph{Set Snares and Traps (Rogue/Warrior, Intelligence, $-1$, 1 skill point)}

A trained character can create and setup traps and snares.  A proficiency check is made at the end of the work period.  A successful check means the trap can work although it's not an indicator that anything will actually trigger the trap.  A failed check makes the character believe the trap will work but it's actually faulty in some way (rope fails to snatch, components are broken, etc.). There is no means of checking a failed trap without attempting to spring it which ruins the trap anyway.

\textit{Small Snare}: With 6d10 minutes worth of work, a snare can be set to capture small game (rabbits, stoats, and other vermin).  Animal snares require little components and can be made from natural materials such as vines.  

\textit{Large Snare}: A large snare, designed to catch up to large-sized creatures (stags, bears, tigers, etc.) can be constructed with the help of at least one other person and 2d4 hours of work at a $-4$ penalty to the check.  Large snares include pitfalls and nets and usually require some kind of tool like a shovel.

\textit{Man Traps}: These are cunning traps specifically rigged to kill intelligent humanoids.  These are usually jury rigged traps made from common materials on hand such as a crossbow attached to a tripwire or spiked springboards.  Usually one person can construct man traps at a -4 penalty and setup time is based on their complexity.

\textit{Special Bonus}: Characters proficient in animal lore gain a +2 bonus when setting snares to catch game.

\index{Spellcraft}\paragraph{Spellcraft (Wizard, Intelligence, $-2$, 1 skill point)}

This skill allows an observer to identify magic and magical effects.  If a person observes a spell as it's being cast, a successful check allows them to identify what the spell is.  A person can also identify a spell based on the components used.  This check can be used to identify the school of a spell already in effect (useful for permanent enchantments) if the effect is observable and obvious.  

\textit{Special}: Specialist wizards gain a +3 bonus to identify spells from their school but a $-3$ penalty to identify forbidden spells.

\index{Survival}\paragraph{Survival (General, Wisdom, +0, 1 skill point, Untrained)}

Survival is the basic ability to survive in hostile terrain.  A survivalist reduces the chance of being lost by 5\% per skill point (only the highest trained character counts).  

\textit{Forage}: A survivalist can forage for food and water, finding enough resources to sustain a number of human-sized characters equal to their margin of success.  Foraging requires an hour and a successful skill check.  The terrain affects the difficulty of foraging.  For example, it's very easy to find food and water in a forest (+4 bonus) but nearly impossible in the desert ($-4$ penalty).  Sometimes it's easier to find food than water or vice versa; for example, you can always find fresh snow to melt in a frozen tundra but finding food is more difficult.

\textit{Fire-Building}: A trained survivalist can automatically build a fire in 2d10 minutes without a tinderbox.  If wet tinder is provided, the time is increased to 3d10 minutes and a successful skill check must be made.  

\textit{Sense Direction}: The character must concentrate for 1d6 rounds.  On a successful skill check they know their relative direction.  On a failed check they misread their direction by 90 degrees east or west and an automatic failure makes them believe they're headed in the opposite direction.  Add a +2 bonus if the sun or moon is visible, +2 if a large landmark is visible (mountain or river), $-1$ at night, and $-2$ in fog or reduced visibility. Determining direction underground is nearly impossible for most creatures (flat $-4$ penalty) although dwarves automatically know their direction underground.  
A character always believes their reading is correct (GM rolls in secret).  Further checks cannot be made until conditions change such as a new landmark appearing or the sun/moon becoming visible. 

\textit{Weather Sense}: With a successful check the character can accurately predict weather conditions (humidity, wind, and temperature spikes) for the next 6 hours.  A failed check gives a false reading.  Only one check can be made every 6 hours.  For every 6 hours of observation, a character gains a +1 bonus to their next skill check as they're able to predict weather patterns.  Sleep or other distracting activities negate this bonus.

\index{Swimming}\paragraph{Swimming (General, Strength, +0, 1 skill point)}

A character with this skill is a practiced swimmer and does not have to perform skill checks except in hazardous conditions.  An untrained swimmer can float and hold their breath but will sink in anything but the calmest waters.  This skill is explained fully in Chapter 9: Adventuring and Exploration

\index{Tracking}\paragraph{Tracking (Rogue/Warrior, Wisdom, +0, 2 skill points)}

This skill functions exactly the same as the ranger class skill.  However, non-rangers automatically suffer a $-6$ penalty to their tracking skill due to its extreme difficulty.  

\textit{Special}: Rangers are automatically proficient in this skill.  Further skill points give them the standard +3 bonus on top of the bonus they receive for being a ranger.

\index{Trades}\paragraph{Trade (Varies)}

Trade skills cover the knowledge, application, and practice of a professional trade or craft.  One skill point gives proficiency in a single trade skill.  This skill allows the supervision of assistants allowing untrained people to assist in the skill check.  This list is nowhere near exhaustive but represents common jobs seen in many societies.

\index{Use Rope}\paragraph{Use Rope (General, Dexterity, +0, 1 skill point, Untrained)}

A trained character has extensive knowledge of knots and tying techniques.  A skill check at a $-6$ penalty can be used to slip free of knotted bonds.  A trained character gains a +2 attack bonus with a lasso and a +2 bonus to climb checks when assisted with a rope.  A trained character can bind someone with rope: on a successful check, no human-sized creature can break the binding unless they have strength 20 or greater.  A failed check allows a character with 18 strength or greater to break the binds or slip out with a dexterity check at $-12$ penalty.

\end{multicols}

\captionof{table}{Trade Skills}\label{performskills}
\noindent
\begin{longtable}{|p{0.12\textwidth}|p{0.58\textwidth}|p{0.23\textwidth}|}
\hline
Trade	& Description	& Notes \\
\hline\hline
\endfirsthead
\multicolumn{3}{c}{\textit{\ldots continued from previous page}} \\
\hline
Trade	& Description	& Notes \\
\hline\hline
\endhead
\hline
\multicolumn{3}{r}{\textit{continued on next page \ldots}}\\
\endfoot
\endlastfoot
\rowcolor[gray]{.9}Agriculture	& Planting, harvesting, irrigation, tending of farm animals, herding, preparation of meat and vegetables.	& General, Intelligence, +0, 1 skill point \\
Armorer*	& Creation of metal armors.  A check is made at the end of the crafting period, a failed check resulting in flawed armor which has an AC penalty of 1 (a person wearing flawed armor is usually unaware of this issue).  Proficient armorers can detect flaws by studying a suit of armor.  Field plate or full plate requires the recipient to be present for a week of production.  An armorer can modify plate mail to fit someone else with a week's worth of work.	& Warrior, Intelligence, $-2$, 2 skill points \\
\rowcolor[gray]{.9}Artist (Field)	& Fundamental theories of art including color theory, values, anatomy, perspective, and form.  An artist specializes in a chosen field (painting, sculpting, etc.) when choosing this proficiency.  An artist gains a +1 bonus when appraising other art or making checks related to music or acting.	& General, Intelligence, +0, 1 skill point \\
Blacksmith*	& The creation of metal implements like horseshoes and crowbars.  Blacksmiths also create blunt metal weapons such as maces and flails.  This skill does not include the creation of bladed weapons or armor (those are armorer and weaponsmith appropriately).	& General, Strength, +0, 1 skill point \\
\rowcolor[gray]{.9}Bowyer/ Fletcher*	& Allows the creation of bows and arrows (a weaponsmith is required to make arrowheads).  Successful checks create a bow that lasts for a long time.  A failed check results in a bow that snaps under stress (a roll of a natural 1 on an attack check) or arrows that have a 50\% chance of automatically missing when fired.  A skilled bowyer can detect flaws in bows and arrows. 	& General, Dexterity, $-1$, 1 skill point \\
Brewing	& A proficient brewer has all the knowledge to create alcoholic drinks including setting up a still, selecting the proper ingredients, and determining if an alcoholic drink is pure or watered down.	& General, Intelligence, +0, 1 skill point \\
\rowcolor[gray]{.9}Carpentry	& The character can construct and repair items made from wood.  Complicated constructions require either plans (usually from an engineer) and delicate or detailed work requires a check.	& General, Strength, +0, 1 skill point \\
Cobbler	& Craft, design, and repair of shoes and footwear.	& General, Dexterity, +0, 1 skill point \\
\rowcolor[gray]{.9}Cooking	& Mastery of the science of cooking.  Most people can prepare simple meals but a character proficient in cooking has knowledge of ingredients, the relation of flavors and tastes, and the different methods of cooking to prepare extravagant meals.	& General, Intelligence, +0, 1 skill point \\
Fishing	& Knowledge of fishing with a wide range of tools including hook and line, spear, or net.  With hook, spear, or hand (in shallow water) a character can catch a number of fish equal to their margin of success (minimum 1) per hour or three times this amount with a fishing net.  The GM should adjust the difficulty in areas with few or abundant fish.  	& General, Wisdom, +0, 1 skill point \\
\rowcolor[gray]{.9}Gem Cutting	& Uncut gems aren't as valuable as a finished product.  A proficient character can finish 1d10 stones in a day on a successful check.  Failed checks ruin the quality of a gem and reduce its value by one step lower than it is.  An automatic success increases the value of the gem by one step.	& Wizard/Rogue, Dexterity, $-2$, 2 skill points \\
Hunting	& Proficiency in stalking game.  It typically takes an hour to pick up a trail and a skill check is made by the GM.  A successful check puts the hunter within 100~+~1d100 yards of his prey.  Additional skill checks may be made to close an additional 20 yards although a failed check in this case automatically alerts the prey to the hunter assuming the prey doesn't have special senses (proper hunting includes masking scents and staying downwind).  A hunter can lead an untrained party at a $-1$ penalty for each untrained individual.  A skilled hunter can stalk prey on a horse although they can't close the distance without losing surprise due to the horse's noise.  Hunting dogs can be used to cut down the time spent hunting and chase prey but the barking of dogs automatically alerts prey after they're stalked.	& Warrior/Rogue, Wisdom, $-1$, 1 skill point \\
\rowcolor[gray]{.9}Mining	& Knowledge of surveying and mining.  A character can survey up to 4 square miles of land per week, making a proficiency check at the end of the week.  A successful check (rolled in secret by the GM) determines what minerals (if any) are located in the area.  A failed check reveals nothing while checks that fail by 10 or more reveal a false finding.  A character with mining skills can also supervise unskilled laborers (5 per point spent) and guide a party through constructed mines with a successful check.	& General, Wisdom, $-3$, 2 skill points \\
Pottery	& The ability to craft items made from clay.  A potter can create four small items (cups), two medium sized items (vase), or one large item (gallon jugs) per day on a successful check which includes firing the items in a kiln.  Raw materials cost 1 copper per small item, 3 copper per medium item, and 5 copper for large items.  This skill only covers the creation of clay items.  Additional details, such as paintings and etchings, are handled by artists. 	& General, Dexterity, $-2$, 1 skill point \\
\rowcolor[gray]{.9}Seamanship	& Training in basic seamanship such as rigging lines, hoisting sails, and damage control.  This skill does not cover actual navigation at sea.  	& General, Dexterity, +1, 1 skill point \\
Seamstress/ Tailor	& Covers embroidery, repair, and design of clothing made from threaded materials such as silk, cloth, burlap, and leather.  Most adventurers can make basic repairs with needle and thread but a seamstress can design new clothing from scratch.   	& General, Dexterity, $-1$, 1 skill point \\
\rowcolor[gray]{.9}Stonemason	& Covers the creation of stone structures, cutting, chiseling, and carving stone.  Stonemasons can supervise unskilled laborers (5 per point spent).  Skill checks are necessary for detailed work.  Dwarves are accomplished stonemasons and have an effective +0 difficulty. & 	General, Strength, $-2$, 1 skill point \\
Tanner	& Knowledge in turning hide into leather.  Tanners may also create leather items such as backpacks, tunics, coats, and leather armor.	& General, Intelligence, +0, 1 skill point \\
\rowcolor[gray]{.9}Weapon\-smith*	& This skill covers the construction of bladed weaponry and crossbows.  Blunt weapons are covered by blacksmiths.  Failed checks waste the raw materials and ruin the weapon.	& Warrior, Intelligence, $-3$, 3 skill points \\
Weaving	& Weavers create thread, tapestries, and draperies from wool or cotton.  A weaver can create two square yards of material per day.  Skill checks are necessary when creating detailed work or using delicate materials like silk.	& Weaving, Intelligence, $-1$, 1 skill point \\
\hline
\end{longtable}
\noindent
\begin{tabular}{p{.98\textwidth}}
*These craft skills follow special rules for creation.  First, convert the cost of the item created into silver pieces (1 gp = 10 sp) if necessary.  Raw materials costing a third of the item's value must be spent.  At the end of a week's work, a skill check is made.  Multiply the margin of success (minimum 1) by 50 sp: this is how much work has been accomplished in a week's time.  When the amount of work equals or exceeds the silver value of the item, it has successfully been constructed.  Doubling, tripling, or quadrupling the value of an item results in it being finished in one half, one third, or one fourth of the time. \\
 \\
Sometimes it's necessary to find out how much work is accomplished in a single day.  Convert the items value into copper pieces (1 gp = 10 sp = 100 cp) and instead multiply the margin of success by 50 cp. \\
 \\ 
Unless otherwise noted, a failed check results in no progress made that week and wastes the raw materials. An automatic failure ruins the item wholly. \\
\end{tabular}\vspace{.5em}
