\chapter{ALIGNMENT}

\begin{multicols}{2}

\index{Alignment}Alignment is a guideline that influences a character's personality and outlook on morality, society, and life in general, and the influences of the universe as a whole.  Alignment is both a tool for role-playing and a mechanic for certain spells and abilities.  Alignment is always based on a character's actions, never thoughts; a person is not evil simply for thinking evil thoughts.  Characters are not bound to their alignment; people can act irrationally, perceive they're acting properly, or change beliefs.

Alignment is divided between good vs. evil and law vs. chaos, which defines the universe and its people as a whole.  There are nine alignments as a result; lawful good, neutral good, chaotic good, lawful neutral, neutral, chaotic neutral, lawful evil, neutral evil, chaotic evil.

\section{LAW VS. CHAOS}

Law and chaos reflect a character's outlook towards society and personal relationships.

\index{Law}Lawful creatures respect authority, organizations, and rules as imperative to the proper function of life.  Concepts such as family, tradition, honor, and loyalty are of the highest importance.  The existence between society and government is natural and the foundation of strong organizations such as guilds and churches are necessary to maintain order.  Without law, society has no means to trust or depend on each other and falls prey to encroaching chaos.  While lawful creatures can be pragmatic, they follow rules and tradition even if they're inefficient or draconic.

\index{Chaos}Chaotic creatures believe that nothing in the universe is certain and that individual is responsible for his own lot in life.  Destiny is uncertain, events in the universe aren't correlated, and one must adapt to their current situation to survive.  Chaotic creatures believe rules and tradition stifle progression.  While chaotic creatures aren't necessarily anarchistic, they tend to place the needs of themselves or personal acquaintances over society as a whole.

Neutrality in respect to law and chaos is a belief that all things must have a counterpart.  Neutral creatures believe that, for every action there's an equal and opposite reaction and without this the universe will fall apart.  Neutral creatures do what they think has the most advantageous results whether this means obeying law in one situation then embracing chaos in the next.

Law and chaos are always in respect to society, authority, or the majority.  Following a rigid personal code doesn't make a character lawful.  Chaotic creatures can be honest and mindful of authority.  A barbarian raider who refrains from harming women or children is likely chaotic.  An assassin who follows the creed of his local guild is likely lawful.

\section{GOOD VS. EVIL}

Good and evil reflects a character's outlook on morality; his personal moral compass deciding what's right and wrong.

\index{Good}Good creatures respect the value of life and sacrifice personal resources to help each other.  They believe in protecting the weak, defending the innocent, and promoting these values so that others will do the same.  Good is not absolute, thus few people are good all the time, but good characters reflect on their failings and strive to overcome them.  

\index{Evil}Evil creatures are destructive, self-serving, and sacrifice others to achieve their personal goals.  Evil can be subtle or overt and any obstruction to an evil creature's goals is a hindrance that must be exploited or removed.  Few creatures are consistently evil or even recognize themselves as evil, thus an evil creature can be unaware of the results of their actions.

Neutrality in respect to good and evil reflects a tendency to not judge actions based on the moral outcome.  Neutral creatures aren't particularly cruel or destructive but they feel no obligation to spend personal resources helping others.  As a result, neutral creatures form bonds based on personal relationships not an obligation to serve or force others into service.  \index{Alignment!Animals}Some creatures, particularly instinctual ones like animals, have no moral compass.  Animals kill because they're hungry, not out of cruelty.  

Good and evil are defined by society.  What one society deems to be good another may believe is evil.  As a result, few mortal creatures are truly evil.  Regardless, good is always constructive while evil is always destructive.  Raiding villages for supplies and slaves might be a common practice in one civilization (albeit not a good one) but razing the village to the ground and massacring its residents for personal enjoyment is evil.  

\section{THE NINE ALIGNMENTS}

There are nine alignments that govern all creatures.  Most mortals are born neutral and their personal upbringing influences their final alignment.  Outsiders and creatures from other planes are created from the raw energies of that plane.  While they can change alignment, planar creatures are always born with an alignment that matches their home plane and changing to a different or opposite alignment is rare.

\paragraph{Lawful Good:} The belief that a strong, orderly society can improve life for the majority of people.  The law is designed to protect and guide the people and the people are expected to uphold the law for the mutual good of all.  Lawful good creatures are honest and forthright, upholding the greatest good while causing the least harm.  If restrictions to personal freedom must be made to ensure the greatest good for everyone, then it's a necessary sacrifice so long as no undue harm is wrought.  Considered to be good aligned.

\paragraph{Neutral Good:}  The belief that either societal extreme isn't necessarily the correct path to maintaining the good of all creatures.  If a strong government protects the people then it should be supported.  If abandoning tradition liberates an oppressed group, then it must be done.  Social structure and personal freedom are but tools for achieving goodness and no belief is above another if the end result is the same.  Considered to be good aligned.

\paragraph{Chaotic Good:}  The belief that personal freedom allows unrestricted attempts to achieve the greatest good.  Chaotic good creatures believe that rules, even ones with good intentions, can be abused and that people should follow their own moral compass, not a mandate.  Creatures of this alignment tend to be freedom fighters and detest bullies who hide behind the law.  While benevolent and forthright, this belief clashes with good order and it's quite possible for people with radically different opinions on achieving good to violently clash.  Considered to be good aligned.

\paragraph{Lawful Neutral:} The belief that law and order are paramount to a structured society.  The universe is cruel and moral extremes do nothing but hamper the progression of society as a whole.  Lawful neutral creatures staunchly believe in order, tradition, and a strong government.  Personal freedom and individuality are the price for order and balance.  Considered to be neutrally aligned.

\paragraph{True Neutral:} The belief that the forces of the universe must be kept in balance.  Laws can be restrictive and easily abused while chaos can be uncontrollable and unpredictable.  Goodness can be counter-productive while evil is unnecessarily destructive.  True neutral characters are pragmatic and judge actions, not appearances.  They often side with factions that have the most compelling argument but are prone to defecting when one becomes too powerful.  Most neutral characters lean towards a particular alignment; those who believe in complete neutrality are uncommon.  Considered to be neutrally aligned.

\index{Non-aligned Creatures}Creatures with animal-like intelligence or no intelligence (a construct or ant) are always neutral because they have no understanding of morality.  A dog can be trained to kill but they cannot comprehend the results of their actions.

\paragraph{Chaotic Neutral:}  The belief that there is no order in the universe and that dwelling on moral extremes, good or evil, inhibit personal freedom.  Chaotic neutral characters do whatever they feel like with little thought towards the results but neither are they completely cruel or uncalculating.  A chaotic neutral creature may simply be impulsive to a fault, make sudden decisions, or prefer to live their lives by the flip of a coin.  Considered to be neutrally aligned.

\paragraph{Lawful Evil:} The belief that laws and society can be abused for mutual gain.  Society creates a definitive gap between those with power and those without; the rules should be made to keep those without power from acquiring it.  If a law benefits the creature while hurting someone else, it should be exploited.  Laws are obeyed only out of fear of punishment or reprisal.  Considered to be evil aligned.
 
\paragraph{Neutral Evil:}  The belief that evil is inhibited by the structure of laws and wasted in a truly chaotic system.  Neutral evil creatures are amoral, avaricious, selfish, and untrustworthy.  They'll work with others if it works in their favor and betray allies if the benefits outweigh the risks.  Their allegiance is measured by personal gain and ``friends'' are only kept as close as the reach of their longest weapon.  Considered to be evil aligned.

\paragraph{Chaotic Evil:} The belief that there is no meaning in the universe except personal pleasure.  The weak are to be exploited, and the strong dominate the weak.  Laws are pathetic tools designed to protect those who pretend to have power.  They care not for meaning or reason to their actions except for the end results.  Chaotic evil creatures are nearly impossible to assemble in any group for an extended length of time unless their desires are consistently satiated.  Considered to be evil aligned.

\section{TRACKING AND CHANGING ALIGNMENT}

\index{Alignment!Changing Alignment}After character creation, the player is no longer in charge of their alignment.  Each session, the GM charts a player's decisions and decides where they fall on the alignment scale.  Most minor actions are inconsequential; people act against character all the time.  Major actions should be taken into consideration.  If a character severely acts against alignment then the GM has the right to change it.  It's recommended to tell players about to shift to reflect on their past decisions before ultimately changing their alignment.  Changing a player's alignment requires careful consideration and there's no harm in ignoring the rules for it.

If a person's alignment changes due to conscious, willing decisions they require double the experience points to level up to the next level.  If a person's alignment changes more than once in a single level they lose all experience points gained that level and have to earn double the experience points to level up.  

A character whose alignment changes involuntarily (such as through magic) are not penalized but they gain no further experience points until they revert to their original alignment.  For example, a paladin dominated into doing evil acts must atone back to lawful good in order to gain experience points again.  If a character decides to remain in their new alignment they require double the experience points to level up.

\end{multicols}

\noindent\includegraphics[width=6.75in, height=3.5in]{testblock.pdf} 
