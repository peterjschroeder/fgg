\chapter{TREASURE, MAGICAL ITEMS, AND RESEARCH}

\begin{multicols}{2}

\index{Treasure}Treasure is any item the characters acquire on their journey that has value.  This includes, but is not limited to, portable treasures like coins and art, deeds to land, fine goods (such as silk or furs), magical items, titles, bonds, gems, etc.  The vast majority of treasure is not currently circulating in the common market.  It may be the remnants of a lost golden age, half-forgotten rarities commissioned by wealthy vassals, or the results of experimentation by powerful wizards.  The GM is ultimately responsible for the creation and management of treasure in his game world.

Treasure is typically amassed in hordes.  \index{Animals!Treasure}Unintelligent creatures and those with animal-like intelligence rarely collect treasure.  Treasure found amongst these creatures might be the possessions  of their fallen prey. It will be scattered amongst the remains and tossed aside when they clean their lairs.  Some creatures, however, are naturally attracted to shiny objects and actively collect them.  Nesting animals may horde fine silks and leathers for their nests, and voracious creatures with no fine manipulating digits (like sharks) may accidentally consume valuables.  Intelligent creatures have their own motives for hording (or spending) treasure.  As a general rule, intelligent creatures only carry what they need at the time, leaving the rest in homes or lairs.

Treasure can be found randomly, or its discovery can be planned.  When a GM rolls for a random encounter, he also rolls for what treasure that creature might have.  A monster's listing indicates treasure that's carried with them, and what's found in their lairs (again, intelligent creatures will carry usable items and leave behind burdensome wealth).  Planned treasure is designed by the GM to entice or reward the players.  Random treasure should rarely be worth more than planned treasure, as it creates a dependency on random encounters.  If, however, the PCs enjoy hunting or stalking monsters for their wealth, the GM is encouraged to build an adventure around it.

\section{MAGICAL ITEMS}

\index{Magical Items}Magical items almost always appear as mundane equipment.  Bards, sages, and powerful divination magic can be employed to fully understand an item's functions, but very few items will demonstrate their powers until commands are spoken or special conditions are met.  In many cases, discovering the power of a magic item comes purely through experimentation, which may very well bring disastrous results to an unsuspecting party.

\index{Cursed Items}Some magical items are cursed.  Like normal magical items, they almost always appear as normal equipment, and they do not reveal their accursed nature even if divination magic is used.  Some accursed items will function normally most of the time; their curse only activates when the user is in a dangerous or life threatening situation (usually combat).  Once a curse is activated, the item thereafter remains with its owner, magically appearing in his possession if thrown away or discarded, until a \textit{remove curse} or other spell is employed to break the curse.  Knowingly selling or giving away accursed items is unquestionably not a good act, and doing so with particularly potent accursed items (especially those that kill or maim) crosses into the realm of evil.

There is little to no market in magical items; only the strongest and bravest explorers find them regularly, only the wealthiest patrons can afford them, and most people value them more than any sum of money.  A +1 sword could be passed down to the strongest son of each generation, becoming an invaluable family heirloom.  A lord could be given a rod to symbolize his right to rule and protect him from danger, the rod being a greater symbol of his right to rule than the crown jewels itself.  Wizards spend months crafting magical items to gain advantage over their competition, making it foolish to sell or trade away their creations.  Merchants who deal in magical items would draw great attention, requiring great wealth to be spent on taxes and security, thus minimizing profits.  All these factors make the existence of ``magic shops" illogical in the average world.

The monetary value of magical items is therefore up to the GM.  A good rule of thumb is the item's XP value multiplied by 100 gold pieces, but an item's worth is based entirely on the perception of the owner and the potential buyer, and doesn't always have to be represented in liquid cash.  For example, a widower's late wife left him with a magic wand but massive debt; if the PCs get the collectors off his back, he may give up the wand. As another example, a powerful merchant lord may offer a magic suit of armor for bringing back the pelt of a rare golden lion (which could also be made into magic leather armor).  Likewise, the player characters shouldn't expect to earn their weight in gold coins for the sale of magical items, because few people have this kind of wealth.  A vassal may give 500 acres of land and a small keep in exchange for a particularly powerful magic item.  The PCs can also choose to give their extra magical items to henchmen or hirelings, indirectly increasing their own power and inspiring fierce loyalty in the followers in return.

 
\subsection{USING MAGICAL ITEMS}

Activating a magic item's power usually takes an entire round, whether it's done through speaking a command or striking with a weapon.  Some powers happen automatically when certain conditions are met.  Wands, rods, staves, and miscellaneous magical items will either confer their powers automatically or through the use of commands.  Once the commands are known and the item's powers are discovered, the owner can then activate the item at will.

\index{Armor!Magical Armor}\index{Weapons!Magical Weapons}Magic weapons and armor will always confer their bonuses when used or worn, and a character will automatically realize this through combat.  Special abilities contained within magical weapons and armor must be discovered, sometimes through constant use (as the case with attacking certain enemies or taking damage) or through research.  Once the weapon or armor's special power is known, the wielder can activate it at will.

There's a limit to the number of magical items a person can wear and benefit from.  If this limit is exceeded, such as by wearing more than two rings, all magic equipment worn on that part of the body ceases to function until the excess is removed.  If magical bracers are worn, a character cannot also wear a magical phylactery or talisman upon his arm, and only one pair of bracers can be worn at a time or neither will function.  If magical armor is worn, a magical robe cannot also be worn.  A magical hat cannot be worn with a magical helm, etc.  Unless otherwise stated, if an item comes in pairs (such as boots or gloves), matching items must be worn to gain the desired effect. Mismatching magical items (especially those worn on or over the eyes) may have unexpected results.  The following table lists all commonly known types of magical items that must be worn to function.

\index{Magical Items!Usage Limits}\paragraph{Arms:} 1 pair of bracers, 1 phylactery or 1 talisman.

\paragraph{Body:} 1 suit of armor or 1 robe.

\paragraph{Clothing:} 1 amulet, 1 brooch, 1 periapt, 1 scarab or 1 talisman.

\paragraph{Face:} 1 pair of eyes or 1 pair of lenses.

\paragraph{Feet:} 1 pair of boots or 1 pair of slippers.

\paragraph{Fingers:} 2 rings (one on each hand).

\paragraph{Hands (combines with fingers):} 1 pair of gauntlets or 1 pair of gloves.

\paragraph{Head:} 1 hat, 1 helm, 1 phylactery or 1 talisman.

\paragraph{Leg:} 1 phylactery or 1 talisman.

\paragraph{Neck:} 1 amulet, 1 medallion, 1 necklace, 1 periapt, 1 scarab or 1 talisman.

\paragraph{Shoulders (combines with body):} 1 cloak.

\paragraph{Waist:} 2 girdles.

\vspace{1em}
 
The GM may allow any sort of magical items in his campaign world.  The following table lists some of the possible new types and on what part of the body they are worn.

\paragraph{Arms:} 1 armband or pair of armbands, 1 bracelet.

\paragraph{Clothing:} 1 badge, 1 medal.

\paragraph{Face:} 1 pair of goggles, 1 mask, 1 pair of spectacles.

\paragraph{Feet:} 1 pair of sandals, 1 pair of shoes.

\paragraph{Head:} 1 circlet, 1 crown, 1 headband.

\paragraph{Neck:} 1 collar, 1 medal, 1 pendant, 1 scarf, 1 torque.

\paragraph{Shoulders (combines with body):} 1 cape, 1 mantle, 1 shawl.

\paragraph{Torso (combines with body):} 1 shirt, 1 tunic, 1 vest or set of vestments.

\paragraph{Waist:} 2 belts (interchangeable with girdles).

\subsection{RESEARCHING AND CRAFTING MAGICAL ITEMS}

\index{Magical Items!Crafting}Spell casters gain the ability to craft magical items.  Wizards can scribe scrolls and brew potions at 7\textsuperscript{th} level and craft magical items at 11\textsuperscript{th} level.  Priests can scribe scrolls at 7\textsuperscript{th} level, brew potions at 9\textsuperscript{th} level, and create other magical items at 11\textsuperscript{th} level. 
 
Like research, the crafting of a magical item is a shared activity between player and GM.  The player describes what item he sets out to create and the GM determines the components and required cost.  At no point should the player be aware of the necessary components; that's an element saved for role-play.  Usually, a laboratory is needed for wizards and a powerful holy site is needed for priests; however, scrolls can be scribed in any comfortable, safe location. 
 
Scribing scrolls is a simple task, taking only a few days, and the components shouldn't be more expensive than if a wizard were to scribe the spell into his spell book.  Potions are about double the cost of scrolls, but usually take less time to brew.  Magical items require the most supervision, taking weeks or possibly months to create. They require rare and costly components (which may warrant additional adventuring to acquire) and certain high level spells (or scrolls containing such spells).
 
The necessary components to craft a magical item can be mundane or fantastical, but they should reflect the item being created.  Easily crafted magical items, such as wands and potions, should require more mundane components.  A \textit{potion of invisibility} might be brewed from the sweat of a high level thief, combined with the tail of a chameleon.  More powerful items, like weapons and rings, should require fantastic or seemingly impossible to acquire components.  A mace of disruption might be forged from smearing the ectoplasm of a good aligned ghost on a mace used to destroy 99 undead.  There could very well be a market in the more mundane components amongst alchemists and hedge wizards, but to locate the rare items that are needed for the more powerful magical items will almost always require dangerous quests and journeys through previously unexplored territories.
 
 
\subsection{SCRIBING SCROLLS}

\index{Scrolls!Scribing}To scribe a scroll, the spell caster must know the spell.  A wizard must have the spell in his spell book, and a priest must have access to the spell's sphere.  Wizards can only scribe protection scrolls featuring spells found in the schools they have access to (\textit{protection from elementals}, \textit{petrification}, etc.), but priests can scribe any type of protection scroll.   \textit{Read magic} is not required to understand a scroll that a character has personally scribed.  Three materials are required when scribing scrolls: the quill, paper, and ink.

\paragraph{Quill:} The quill used to scribe the scroll must be fresh and unused.  After scribing a scroll, the magic clings to the quill, making it useless for future scribing.  Higher-level spells should require more extraordinary quills, such as those from a giant eagle, roc, or cockatrice.  Quills can be purchased at the GM's discretion, but handpicking a quill provides a +5\% bonus to the chance of successfully scribing the scroll.

\paragraph{Paper:} The material to be used to hold the magic of the scroll has an effect on the chance for successfully scribing it.  Vellum costs the most but is greatly prized for this purpose, providing a +10\% bonus to the chance of success.  Paper is the next most valuable material to use, providing a +5\% bonus.  Parchment is of average quality and offers no modifiers.  Papyrus is the poorest quality and applies a $-5$\% penalty to the chance to success.

\paragraph{Ink:} The GM determines the components required to create the ink, choosing those that reflect the spell being scribed and costing no more than if a spell caster was hired to cast the spell.  The ink to scribe one \textit{fireball} spell to a scroll could require an ounce of bat guano mixed with the splinters from the torch of the watch captain.  After the necessary components are gathered, the spell caster brews them into ink.  The ink should be unique to each transcription, and the caster cannot save the ink, even if it's for the same spell on the same scroll.
 
After the three materials are gathered, the scribing process can begin.  Wizards must be in a comfortable, safe location with their spell books on hand, and priests require an altar or holy site.  Scribing a scroll requires one full day per spell level or six full days per protection scroll.  Spell casters must remain uninterrupted for the duration of the transcription, breaking only for the minimum requirements of food and sleep.  Any prolonged interruption causes the entire process to fail, and the materials are wasted.
 
After each spell is scribed, the GM checks for success.  The base chance for any scroll is 80\%, increased or decreased due to materials.  Additionally, when scribing spells onto scrolls, a $-1$\% penalty is applied for each spell level, and a +1\% bonus is applied for each level of the spell caster.  If the check is equal to or lower than the chance of success, the scroll has been successfully scribed.  On a failed check, the scroll is cursed, usually resulting in the opposite effect (harmful spells targeting allies and beneficial spells targeting enemies), but the GM may assign any curse to the scroll (Refer to Magical Items---Cursed Scrolls).  The spell caster should have no way to know that he failed the check.  A \textit{remove curse} spell will destroy a faulty scroll.
 
The spell's power is based on the level of the spell caster that transcribed it, and should be noted.  A single scroll can hold 2d4~$-$~1 spells, determined by the GM.  The spell caster is never aware of how much space he has left, as scribing a spell, even if it's being duplicated, is an imprecise art.  Cursed spells as a result of failures always occupy the remainder of the scroll.  A cursed spell doesn't curse other spells on the scroll; only that particular spell is cursed.
 
\subsection{BREWING POTIONS}

\index{Potions!Brewing}Spell casters can brew potions if they know the spell it supplants.  Because of their spell list, priests are limited in the potions they can create, but unlike wizards, they can brew the ever-so-useful \textit{potions of healing}.  Potions require mundane ingredients costing between 200--1,000 gp, but like inks for scrolls, potions also require special components that match their function; the more powerful the potion, the more expensive and rare the special ingredients should be.
 
Wizards require a laboratory to brew potions.  The laboratory's equipment, which includes but is not limited to alembics, flasks, retorts, mortar and pestle, beakers, distilling coils, furnaces, and braziers, must cost at least 2,000 gp.  This cost does not include the space or building that actually houses the laboratory. The laboratory must be enclosed to keep out the elements and have a ventilation system to blow out noxious fumes.  Furthermore, the wizard must pay 10\% of his lab's cost each month to replace broken and worn-out equipment.
 
Priests require a specially consecrated altar to brew their potions.  In order to consecrate the altar, the priest must offer a great sacrifice or special service to his deity.  Once he has completed the ceremony, that priest (and that priest only) can use the altar to brew potions with no additional upkeep, except for simple prayers.
 
The spell caster then requires a number of days equal to the cost of the ingredients divided by 100 (2--10 days) to brew the potion.  During this time, the spell caster must remain uninterrupted, breaking only for the minimum of food and sleep.  Any prolonged interruption ruins the potion and materials.  
 
After the final day, the GM makes a check to determine success.  The base chance of success to brew a potion is 70\%.  A $-1$\% penalty is applied for every 100 gp worth of ingredients used, and a +1\% bonus is applied for every two levels of the spell caster (rounded up).  If the check is equal to or lower than the percentage, the brewing process was successful.  If the check fails, the resulting potion is unintentionally cursed, becoming poison or other accursed potion, philter, oil, etc. (GM's choice).   The spell caster doesn't know if his potion has succeeded or failed, though a detect poison spell may help in some instances.
 
Potions should be labeled for future reference, as even equivalent potions have a random assortment of colors and odors.
 
\subsection{CRAFTING MAGICAL ITEMS}

\index{Magical Items!Crafting}Wizards can craft most magical items including weapons, wands, staves, rods, rings, bracers, and cloaks. However, artifacts, relics, intelligent items, books (except spell books) and magical items with a racial connection, such as Elven cloaks or Dwarven war hammers, can only be fashioned by NPCs.  The GM can also restrict the creation of any other magical items, based on his campaign standards, but ultimately players should be allowed to craft their own items provided they go through the complicated and time consuming process of doing so.  Like researching original spells, creating new magical items is a process involving the player writing up what he expects and the GM adjudicating it.  The exact process for enchantment should be solved through role-playing: by seeking ancient lore, asking sages, and contacting otherworldly figures.

\paragraph{Vessel:} All magical items require a vessel to which the magic will be bound.  The vessel must be flawless, made from the finest materials and crafted by a master craftsman.  The item might be embroidered with gold and platinum threads, encrusted with gem dust, made from the strongest of metals, and/or engraved with magical symbols or runes.  As a general rule, the cost of the raw materials should not be less than 100 gp for the smallest and most minor of items to be enchanted, but most vessels have a market value of 1,000 to 10,000 gp. The most powerful magical items require the most rare and hardest to find vessels.  Magical arrows, darts and bolts are usually crafted a dozen or more at a time, and once crafted as a matched set are treated as a single vessel. 

\paragraph{Laboratory:} Once the wizard has procured the vessel, he must find a suitably equipped laboratory or similar workspace to prepare and enchant the vessel until it becomes a magical item.  The lab must be at least 60 feet~$\times$~60 feet or otherwise protected from becoming contaminated from stray magical energy.  Any extraneous magical items or spell casting to come within 30 feet of the vessel after this stage of the items creation will contaminate the enchantment.  Also refer to \textit{enchant an item} spell.  

\paragraph{Prepare the Vessel:} The next step requires material components to prepare the specially crafted vessel.  Magic weapons and armor might need to be rubbed with rare oils to help absorb the magic during the enchantment process.  A magic ring may need to be sprinkled with the dust of a crushed emerald.  Material components should cost at least 500 gp, and more powerful items should require more expensive components.  The preparation process should take a minimum of two weeks to a month and could take much longer for the most powerful items.  If this step is ignored, the item will not hold the enchantment and becomes a non-magical failure.

\paragraph{Enchantment:} Once the vessel has been created and prepared, any spell caster can cast \textit{enchant an item} upon it; even a scroll may be used.  Once \textit{enchant an item} spell is successfully completed (refer to the spell's description), the wizard crafting the magic item has up to 24 hours to begin performing rituals and casting spells into the vessel, or the vessel loses its magical pathways and enchant an item must be cast again.  The wizard can cast the necessary spells from his memory or from scrolls, but not from any other magical items.

The spells and/or rituals required to finalize a magical item should reflect the item's nature and are harmlessly absorbed by it.  Items with powers that have no spell equivalent require a new spell to be researched or must absorb rare magical essences or ephemeral materials that reflect its nature.  A \textit{wand of fireballs} could involve casting \textit{fireballs} into the wand; a \textit{sword of troll slaying} could involve casting multiple \textit{enchanted weapon} spells after the blade soaks in the blood of a troll king for a week.  

Magical items that are described as having charges, such as wands, rods, staves, certain rings, etc. (refer to Appendix II) can never be made permanent.  These magical items hold a number of charges equal to the number of spells cast onto them, up to the maximum number stated for their type of item.

To complete each spell requires that the caster remain in contact with the vessel for 2d4 hours per spell level.  The caster does not know if these spells are successfully absorbed into the vessel or not, so the GM must make a secret save vs. spell for the vessel as if it were the caster, modified by wisdom---mental defense modifier.  The casting of each spell must be begun within 24 hours of the previous spell's completion, regardless of success or failure.  

Spells cast onto a magical item will not operate more than once unless a \textit{permanency} spell is cast on them.  When a wizard casts permanency to finalize a magical item, he only runs a 5\% risk of permanently reducing his constitution by 1 point.  The caster rolls an unmodified 1d20 and loses a point of Constitution only on roll of a natural one.  \textit{Permanency} is used to create magical items such as rings, armor, and weapons.  

After the enchantment is finalized, the GM rolls a check.  The base chance of success is 60\%.  A +1\% bonus is added per level of the caster, and a $-1$\% penalty is subtracted for each special process, unusual spell or unique ingredient used.  The GM is free to add bonuses if the PC takes precautions (such as buying specialized equipment) or follows well-known procedures, and to subtract penalties if the PC skips seemingly unimportant details or is unprepared in some way.  If the check is equal to or lower than the percentage, creation is successful. 

Over 90\% of all successful magical items are standard and act just as described in their magical item description (refer to Appendix II, and to the \textit{dispel magic} spell).  Most wands are caster level 6, most staves are caster level 8, and most other items are level 12, etc. However, when a magical item is finalized, the caster rolls 1d10.  This is the percentage chance (1--10\%) that the caster has created a flawless "archetype" or perfect specimen of that magical item.  Roll 1d100 to determine if an archetype has been created.  An archetype magical item has a caster level and spell intensity equal to the caster who created it +3 (it will always be caster level 14 or higher).  It may have additional properties not found in the standard description, at the GMs option.  An archetype magical item is found in treasure only 1\% of the time (00 on 1d100), and even then, it is only found in the hands of the one who created it, or worse yet, something powerful enough to take it from the one who did. 

On a failed roll, the item is unknowingly cursed.  An accursed item acts in a way opposite its intended manner (accursed weapons lower chances to hit, wands blow up in the caster's face, etc.).  The wizard doesn't know the creation was unsuccessful.  An accursed item cannot be created by design.  If this is attempted, the item created is a completely non-magical failure, whether its creation is successful or not.  
 
Priests have an easier time crafting magical items, but the spheres allowed to the priest limit their nature.  After acquiring a specially consecrated, exceptionally crafted vessel, the priest prays over it for two weeks. Another week is spent fasting and meditating followed by a single day of purifying the item.  After these rituals are completed, the priest places the vessel upon a specially prepared altar at a holy site and pleas to his deity and calls upon the divine might of his faith to enchant it.  
 
There's a cumulative 1\% chance each day that the priest's prayers will be heard, and the item enchanted.  The GM rolls once each day, until the priest stops praying or the item is enchanted.  If the item is one with charges, the appropriate spells must then be cast over it, within a 24-hour period.  Once the item is enchanted, it is sanctified and consecrated to seal the magic within.  If any steps in the priest's process are halted or disturbed for anything but a minimum of food and rest, a new vessel must be acquired, and the entire process must be restarted.
 
Because priests only require a vessel and their faith, the PC must be a true follower, and the GM should scrutinize the PC's previous actions and current motives.  Magical items enchanted by priests should also have a direct relation to their faith or be used to help the priest in a coming task.  If the priest has acted against his beliefs or crafted an item that counteracts his tenants, the GM can rule that the creation fails completely or the resulting item is cursed.  Because of this, priests never craft magical items for other people, unless they are also believers, or their actions beneficially influence the faith.
 
\subsection{RECHARGING MAGICAL ITEMS}

\index{Magical Items!Recharging}If an item has charges, it's possible to recharge or add charges.  To recharge an item, it must first be prepared by casting \textit{enchant an item} spell or by having a priest pray and meditate over it as noted above.  When casting \textit{enchant an item} to prepare an item, the normal casting time of the spell is used.  After the item is prepared for enchantment, the spell caster casts the appropriate spells or melds the proper components into the item. 
 
Each time an item is enchanted for a recharge, the spell caster must make a saving throw vs. spell with a $-1$ penalty.  Failure indicates the item crumbles into dust.
 
\subsection{DESTROYING MAGICAL ITEMS}

\index{Magical Items!Destroying}Destroying a magic item is as simple as breaking its vessel.  Attacking a magic item held by an opponent is no more difficult than attacking a mundane item; however, it receives a bonus to its saving throw equal to the power of the item (+1 to +5).  Some magical items have specific effects when destroyed, but GMs are encouraged to be creative with other items.  A weak magic item may create a visible flash of light or a strong, distinctive odor, while more powerful items could explode in a burst of magical energies.

\end{multicols}

\noindent\includegraphics[width=6.75in, height=3in]{testblock.pdf} 

